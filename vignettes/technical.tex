\documentclass[12pt]{article}
\usepackage{listings}
\usepackage{Sweave}
\begin{document}
\Sconcordance{concordance:technical.tex:technical.Rnw:%
1 2 1 1 0 21 1}

\section*{Profitability}
Profitability is composed of six variables: gross profits over assets ($GPOA$), return on equity ($ROE$), return on assets ($ROA$), cash flow over assets ($CFOA$), gross margin ($GMAR$), and accruals ($ACC$). $GPOA$ is calculated as gross profits ($GPROF$) over total assets ($TA$). $$GPOA \ = \ \frac{GPROF}{TA}$$ $ROE$ is calculated as net income ($NI$) over book equity ($BE$), which is shareholders' equity (the difference of Total Liabilities and Shareholders' Equity ($TLSE$) with Total Liabilities ($TL$)) - preferred stock (the sum of redeemable preferred stock ($RPS$) and non redeemable preferred stock ($NRPS$)). $$ROE \ = \ \frac{NI}{BE}$$ $ROA$ is calculated as $NI$ over $TA$. $$ROA \ = \ \frac{NI}{TA}$$ $CFOA$ is calculated as $NI$ + depreciation ($DP.DPL$) - changes in working capital ($CWC$) - capital expenditures ($CX$) all over $TA$. $$CFOA \ = \ \frac{NI \ + \ DP.DPL \ - \ CWC \ - \ CX}{TA}$$ $GMAR$ is calculated as $GPROF$ over total revenue ($TREV$). $$GMAR \ = \ \frac{GPROF}{TREV}$$ Finally, $ACC$ is calculated as $DP.DPL$ - $CWC$ all over $TA$. $$ACC \ = \ \frac{DP.DPL \ - \ CWC}{TA}$$ We then standardize all components of profitability to z-scores and then standardize all profitability scores into z-scores. $$Profitability \ = \ z(z_{gpoa} \ + \ z_{roe} \ + \ z_{roa} \ + \ z_{cfoa} \ + \ z_{gmar} \ + \ z_{acc})$$
\section*{Growth}
Growth is measured by differences in profitability across a time span of four years. Though AQR recommends measuring growth across a time span of five years, public information that is both consistent and well-organized in 10-K forms is only available for a time span of four years, and it is still too early in the most recent year (2015) for most companies to have submitted a 10-K form. Thus, we measure growth using a time span of four years, which we will update once this year's 10-K form is submitted for each company in the Russell 3000 Index. As of now, $$Growth \ = \ z(z_{\Delta gpoa_{t,t-4}} \ + \ z_{\Delta roe_{t,t-4}} \ + \ z_{\Delta roa_{t,t-4}} \ + \ z_{\Delta cfoa_{t,t-4}} \ + \ z_{\Delta gmar_{t,t-4}} \ + \ z_{\Delta acc_{t,t-4}})$$
\section*{Safety}
Safety is composed of six variables: beta ($BAB$), idiosyncratic volatility ($IVOL$), leverage ($LEV$), Ohlson's O ($O$), Altman's Z ($Z$), and earnings volatility ($EVOL$). $BAB$ is calculated as the negative covariance of each company's daily price returns ($pret_{c_i}$) relative to the benchmark daily market price returns ($pret_{mkt}$), in this case the S\&P 500, over the variance of $pret_{mkt}$. $$BAB \ = \ \frac{-cov(pret_{c_i},pret_{mkt})}{var(pret_{mkt})}$$ $IVOL$ is the standard deviation of daily beta-adjusted excess returns. In other words, $IVOL$ is found by running a regression on each company's price returns and the benchmark, then taking the standard deviation of the residuals. Leverage is -(total debt ($TD$) over $TA$). $$Leverage \ = \ -\frac{TD}{TA}$$ 
\\
\\
\\
\\
\\
\\
\\
$$ O \ = \ -(-1.32 \ - \ 0.407 \ * \ log\left(\frac{ADJASSET}{CPI}\right) \ + \ 6.03 \ * \ TLTA \ - \ 1.43 \ * \ WCTA$$
$$ + \ 0.076 \ * \ CLCA \ - \ 1.72 \ * \ OENEG \ - \ 2.37 \ * \ NITA \ - \ 1.83 \ * \ FUTL$$
$$ + \ 0.285 \ * \ INTWO \ - \ 0.521 \ * \ CHIN)$$ 
$ADJASSET$ is adjusted total assets, which is $TA$ + 0.1 * (market equity ($ME$, calculated as average price per share for the most recent year * total number of shares outstanding ($TCSO$) - $BE$)). $$ADJASSET \ = \ TA \ + \ 0.1 \ * \ (ME \ - \ BE)$$ $CPI$, the consumer price index, is assumed to be 100, since we only care about the most recent year. $TLTA$ is book value of debt ($BD$, calculated as $TD$ - minority interest ($MI$) - ($RPS$ + $NRPS$)) over $ADJASSET$. $$TLTA \ = \ \frac{BD}{ADJASSET}$$ $WCTA$ is current assets ($TCA$) - current liabilities ($TCL$) over $TA$. $$WCTA \ = \ \frac{TCA - TCL}{TA}$$ $CLCA$ is $TCL$ over $TCA$. $$ CLCA \ = \ \frac{TCL}{TCA}$$ $OENEG$ is a dummy variable that is 1 if total liabilities ($TL$) is greater than $TA$. $$ OENEG \ = \ TL > TA $$ $NITA$ is $NI$ over $TA$. $$NITA \ = \ \frac{NI}{TA}$$ $FUTL$ is income before taxes ($IBT$) over $TL$. $$FUTL \ = \ \frac{IBT}{TL}$$ $INTWO$ is another dummy variable that is 1 if $NI$ for the current year and $NI$ for the previous year are both negative. $$INTWO \ = \ MAX(NI_t,NI_{t-1}) < 0$$ $CHIN$ is $NI$ for the current year - $NI$ for the previous year all over the sum of the absolute value of $NI$ for the current year and the absolute value of $NI$ for the previous year $$CHIN \ = \ \frac{NI_t \ - \ NI_{t-1}}{|NI_t| + |NI_{t-1}|}$$ Altman's Z is calculated using weighted averages of working capital ($WC$, calculated as $TCA$ - $TCL$), $$WC \ = \ TCA \ - \ TCL$$ retained earnings ($RE$, calculated as $NI$ - dividends per share ($DIVC$) * $TCSO$), $$RE \ = \ NI \ - \ DIVC \ * \ TCSO$$ earnings before interest and taxes ($EBIT$, calculated as $NI$ - Discontinued Operations($DO$) + ($IBT$ - income after tax ($IAT$)) + interest expense ($NINT$)), $$ EBIT \ = \ NI \ - \ DO \ + \ (IBT \ - \ IAT) \ + \ NINT $$ $ME$, and $TREV$, all over $TA$. $$Z \ = \frac{\ 1.2 \ * \ WC \ + \ 1.4 \ * \ RE \ + \ 3.3 \ * \ EBIT \ + \ 0.6 \ * \ ME \ + TREV}{TA}$$ $EBIT$ is likely an overestimate for a given company due to potentially missing information. $EVOL$ is calculated as the standard deviation of $ROE$ for a four year span. AQR recommends the past five years, but for the same reason stated in the Growth section, we use a four year span. $$EVOL = \sigma\left(\sum_{i=t-4}^{t}ROE_i\right)$$ Likewise, we standardize each variable and then standardize each safety measure, so $$Safety \ = \ z(z_{bab} \ + \ z_{ivol} \ + \ z_{lev} \ + \ z_{o} \ + \ z_{z} \ + \ z_{evol})$$ 
\section*{Payouts}
Payouts is composed of three variables: net equity issuance ($EISS$), net debt issuance ($DISS$), and total net payout over profits ($NPOP$). $EISS$ is calculated as the negative log of the ratio of $TCSO$ of the most recent year and $TCSO$ of the previous year. $$EISS \ = \ -log\left(\frac{TCSO_t}{TCSO_{t-1}}\right)$$ Though AQR uses split-adjusted number of shares, we are currently using $TCSO$ given available information and will adjust for splits in future iterations of qmj. $DISS$ is calculated as the negative log of the ratio of $TD$ of the most recent year and $TD$ of the previous year. $$DISS \ = \ -log\left(\frac{TD_t}{TD_{t-1}}\right)$$ $NPOP$ is calculated as $NI$ - $\Delta BE$ over a four year span all over sum of $GPROF$ for the past four years (for the same reason as explained in the Growth section). $$NPOP \ = \ \frac{NI - \Delta BE}{\sum_{i=t-4}^{t}GPROF_i}$$
\end{document}
